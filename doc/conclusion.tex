%!TEX root = ttc14-fixml.tex

\section{Conclusion}
\label{sec:Conclusion}

In this paper we have presented a \SIGMA solution for the \TTC \FIXML case study.
It covers the core problem of generating Java, \Csharp and C++ code from \FIXML messages including solutions for all of the case study extensions.
The implementation is based on a systematic transformations of XML:
\begin{inparaenum}[(1)]
	\item a text-to-model transformation of an XML input into an XML meta-model,
	\item a model-to-model transformation of the XML meta-model into a object-oriented language-agnostic meta-model called ObjLang, and finally
	\item a model-to-text transformation of ObjLang into a set of source code implementations for Java, \Csharp, C++ and also C (in the second case study extension).
\end{inparaenum}
We have specially opted for using a non-trivial ObjLang model in order to demonstrate a complex, yet expressive and quite concise transformation.
Moreover, the ObjLang with the M2T generators provides a generic object-oriented language model that could be likely reused in other model driven engineering scenarios.

In the third case study extension, we have implemented a generic \FIXML Schema transformation into ObjLang model based on transforming annotated Java classes generated by Java Architecture for XML Binding.