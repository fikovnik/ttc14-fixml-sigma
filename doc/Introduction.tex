%!TEX root = ttc14-fixml.tex

\section{Introduction}
\label{sec:Introduction}

%% Overview
In this paper we describe the solution for the \TTC \FIXML case study~\cite{Lano2014} using the \SIGMA internal DSLs~\cite{Krikava2014}.
The case study involves \emph{text-to-model} (T2M), \emph{model-to-model} (M2M) and \emph{model-to-text} (M2T) transformations, generating Java, \Csharp, C++ code from a \FIXML XML messages.
Furthermore, we describe our approach to the three proposed extensions.
The complete solution is available on a Github\footnote{\url{https://github.com/fikovnik/ttc14-fixml-sigma}} as well as in the SHARE\footnote{\url{http://is.ieis.tue.nl/staff/pvgorp/share/}} environment in the virtual machine image \texttt{Ubuntu12LTS\_TTC14\_64bit\_SIGMA.vdi}.

%% SIGMA
The solution is developed in \SIGMA, a family of Scala\footnote{\url{http://scala-lang.org/}} internal DSLs for model manipulation tasks such as model validation and model transformations.
Developed as an open source project hosted on Github~\footnote{\url{https://fikovnik.github.io/Sigma}},
\SIGMA is a library that provides a dedicated Scala API allowing to manipulate models using high-level constructs similar to ones found in the external model manipulation DSLs such as ETL~\cite{Kolovos2008a} or ATL~\cite{Jouault2006}.
The intent is to provide an approach that developers can use to implement many of the practical model manipulations within a familiar environment, reduced learning overhead and improved usability.

The solution uses the \emph{Eclipse Modeling Framework} (EMF)~\cite{EMF}, which is a popular meta-modeling framework widely used in both academia and industry.
It is directly supported by \SIGMA, however, other meta-modeling frameworks could be used as well, since \SIGMA transformations are technologically agnostic.

%% Organization
The remainder of this document is organized as follows:
\begin{inparaitem}[]
	\item Section~\ref{sec:SigmaOverview} gives a brief overview of \SIGMA.
	\item Section~\ref{sec:SolutionDescription} describes the solution for the case study core problem.
	\item Section~\ref{sec:Extensions} presents the solutions for the three case study extensions.
	\item Section~\ref{sec:Evaluation} evaluates the solution using the evaluation criteria proposed in the case study, and finally Section~\ref{sec:Conclusion} concludes the paper.
\end{inparaitem}
%
% Additionally, two appendixes are provided.
% \begin{inparaitem}[]
% 	\item Appendix~\ref{sec:Configuration} describes the configuration of the SHARE environment that has been done in order to run the \SIGMA solution and
% 	\item Appendix~\ref{sec:Example} shows the generated Java, \Csharp, C++ and C code resulting from running the solution for the example \texttt{test2.xml} \FIXML message.
% \end{inparaitem}