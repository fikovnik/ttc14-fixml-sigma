%!TEX root = ttc14-fixml.tex

\section{Extensions}
\label{sec:Extensions}

In this section we describe our solutions to the three proposed case study extensions.

\subsection{Extension 1 - Selection of Appropriate Data Types}
\label{sec:Extension1}

The source concerning this first extension is located in the \href{https://github.com/fikovnik/ttc14-fixml-sigma/tree/master/ttc14-fixml-extension-1}{\texttt{ttc14-fixml-extension-1}} directory.

\bigskip

There are three changes needed in order to implement this extension.
\begin{itemize}[(1)]
	\item The first one is in the ObjLang meta-model where we need to add new expression classes representing literals for the new data types.
	In this extension we consider the following new data types: \Scala{double}, \Scala{long} and \Scala{integer}.
	While this list does not cover all the possible XML Schema data types, it provides a good basis for demonstrating how some support for additional ones could be added.

	\item The second one concerns the M2M transformation.
	We have to add the necessary support for guessing the data type of an attribute based on the string values of all of the same-tag siblings that have the attribute in question.
	Following is the code snippet that realizes it:
	%
	\begin{scalacode}
	  // basic types
	  val DTString = DataType(name = "string")
	  val DTDouble = DataType(name = "double")
	  val DTLong = DataType(name = "long")
	  val DTInteger = DataType(name = "int")

	  // it also stores the promotion ordering from right to left
	  val Builtins = Seq(DTString, DTDouble, DTLong, DTInteger)

	  private val PDouble = """([+-]?\d+.\d+)""".r
	  private val PInteger = """([+-]?\d+)""".r

	  def guessDataType(value: String): DataType = value match {
	    case PDouble(_) => DTDouble
	    case PInteger(_) => Try(Integer.parseInt(value)) map (_ => DTInteger) getOrElse (DTLong)
	    case _ => DTString
	  }

	  def guessDataType(values: Seq[String]): DataType =
	    values map guessDataType reduce { (a, b) =>
	      if (Builtins.indexOf(a) < Builtins.indexOf(a)) a else b
	    }
	\end{scalacode}

	\item Finally, we need to update the code transformers to generate the appropriate data types.
	For example, in the C++ generator:
	\begin{scalacode}
override def class2Code(p: DataType) = {
  import XMLMM2ObjLang._
  p match {
    case DTString => "std::string"
    case DTDouble => "double"
    case DTLong => "long"
    case DTInteger => "int"
  }
}		
	\end{scalacode}
\end{itemize}

\subsection{Extension 2 - Extension to Additional Languages}
\label{sec:Extension2}

This extension adds a support for the C language.
Since C is not an object-oriented language, more work have to be done in the code generator part.
It is important to note, that the M2M or T2M transformations have to remain untouched.

Instead of classes, in the case of the C language, we generate C struct declarations with appropriate functions simulating object constructors.
For each ObjLang class we generate a header file with a struct definition, a function for the struct creation and a set of functions representing class constructors.
Following is an example of the C code synthesized from this \FIXML message (for the \Scala|Pty| node):
%
\inputminted[fontsize=\fontsize{8}{8},linenos,numbersep=5pt,frame=lines,framesep=2mm]{xml}{listings/example-for-c-code.xml}
%
\begin{ccode}
#ifndef _Pty_H_
#define _Pty_H_

#include <stdlib.h>

#include "Sub.h"

typedef struct {
  char* _R;
  char* _ID;
  Sub** Sub_objects;
} Pty;

Pty* Pty_new();
Pty* Pty_init_custom(Pty* this, char* _R, char* _ID, Sub** Sub_objects);
Pty* Pty_init(Pty* this);

#endif // _Pty_H_	
\end{ccode}
%
\begin{ccode}
#include "arrays.h"

#include "Pty.h"

Pty* Pty_new() {
  return (Pty*) malloc(sizeof(Pty));
}

Pty* Pty_init_custom(Pty* this, char* _R, char* _ID, Sub** Sub_objects) {
  this->_R = _R;
  this->_ID = _ID;
  this->Sub_objects = Sub_objects;
  return this;
}

Pty* Pty_init(Pty* this) {
  this->_R = "21";
  this->_ID = "OCC";
  this->Sub_objects = (Sub**) new_array(2, Sub_init_custom(Sub_new(), "2", "ZZZ"), NULL);
  return this;
}
\end{ccode}

In order to simplify the code generator, we use a helper function \cinline{void **new_array(int size, ...)} that allows us to initialize arrays using simple expressions, which is not directly supported by C language or by the standard C library.

Despite the fact that C is not object-oriented, our organization of the M2T transformation templates makes the implementation only 36 lines longer (30\%) than the C++ version.
The source concerning the extension 2 solution can be found in the \href{https://github.com/fikovnik/ttc14-fixml-sigma/tree/master/ttc14-fixml-extension-2}{\texttt{ttc14-fixml-extension-2}} directory.

\subsection{Extension 3 - Generic Transformation}
\label{sec:Extension3}

The last extension aims at generic transformation between \FIXML Schema and ObjLang model.
Providing such a generic transformation essentially means creating a generic XML schema to ObjLang transformation.
Such a task is far from being trivial and it would require a significant engineering effort.
Therefore we have decided for an alternative solution in which we transform Java classes generated from an XML schema by the Java Architecture for XML Binding (JAXB) tool\footnote{\url{https://jaxb.java.net/}}.

The advantage of this solution is that the JAXB already does all the hard work of parsing XML schema, resolving the element inheritance, substitutability, data types and others.
The model of Java classes in an object-oriented model and therefore the actual transformation into ObjLang is actually easier than in the case of the \FIXML messages.
Finally, the JAXB can be thought of as a another model transformation and therefore our solution is still within the model-driven engineering domain.

The new input is a location of the \FIXML XSD files and the new transformation workflow consists of the following stages:
\begin{enumerate}[(1)]
	\item XSD to Java sources using JAXB.
	\item Compilation of Java source using regular Java compiler.
	\item Java model to ObjLang transformation.
	\item ObjLang to source transformation.
\end{enumerate}

\bigskip

The implementation involves following changes:
\begin{enumerate}[(1)]
	\item Extending the ObjLang model with a new classifier representing enumerated types.
	\item Extending the ObjLang model with the notion of a simple class inheritance.
	\item Extending the ObjLang model with the notion of abstract classes.
	\item A new M2M transformation between Java class model represented by the Java reflection API and ObjLang model.
	\item Extending the M2T transformation to cover the new ObjLang model concepts.
\end{enumerate}

The new transformation is about 30\% smaller than the one developed in the first extension and arguably less complex.
It also demonstrates \SIGMA support for manipulating different models than EMF, as well as easily reusing existing libraries.
This is typically the kind of benefits that one can get from \SIGMA, \textit{.i.e.}, being able to easily reuse an existing API (JAXB), while developing transformation rules in a domain-specific environment.
The source concerning the extension 3 solution is in the \href{https://github.com/fikovnik/ttc14-fixml-sigma/tree/master/ttc14-fixml-extension-3}{\texttt{ttc14-fixml-extension-3}} directory.

One limitation of the current implementation is that we do not extend the C code generator to support the notion of class inheritance and for abstract classes.
A potential solution is described by Schreiner~\cite{Schreiner1993}.
However it requires to build a lot of supporting infrastructure which is out of the scope of this \TTC case.
Similarly to extension 1, we do not handle all the XML schema data types, but only the ones that appear in the \FIXML schema.
