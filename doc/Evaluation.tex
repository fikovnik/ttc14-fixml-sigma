%!TEX root = ttc14-fixml.tex

\section{Evaluation}
\label{sec:Evaluation}

This section evaluation the proposed solution using the evaluation criteria proposed by the case study description~\cite{Lano2014}.

\paragraph{Complexity.}

\Todo{Complexity}

\paragraph{Accuracy.}
%
The accuracy measures the syntactical correctness of the generated source code and how well does the code represent the \FIXML messages.

The generated code is verified that it can be compiled for for all valid examples provided in the case study.
The compilation does not produce any warnings and does not require any special compiler settings\footnote{The C/C++ \texttt{-fPIC} option is a standard option allowing the object files to be included in libraries, \Cf \url{http://gcc.gnu.org/onlinedocs/gcc/Code-Gen-Options.html\#Code-Gen-Options}}.

We used array representation of the same-tag sibling nodes which improves the quality of the code in comparison to numbered fields as well as it scales better.
By implementing the first exception, we have further improved the usefulness of the generated code by guessing the appropriate type~\ref{sec:Extension1}.
While we do not cover the complete XML Schema type system, the solution is extensible and adding a support for a new data type is just a matter of providing the correct mapping into the corresponding languages.
Finally, we have also implemented the generic \FIXML Schema transformation that should result in a complete representation of \FIXML messages in the different languages.

The case study did not require to provide additional features such as property getters and setters or C/C++ destructors.
As such the usability of the generated code is rather limited.
However, the implementation of these features should be rather straight forward.

\paragraph{Development effort.}

Providing a correct measure of the complete development effort is not easy since by working on the case study we were also improving \SIGMA at the same time.
The solution for the code problem generating only Java code took about 4 person-hours in which about half was spent on designing the right ObjLang model by \emph{trial-and-error}.
Extending the code generator facilities for \Csharp was almost for free as it involves only few lines of code.
The C++ version was ready in less than one person-hour.
The first extension was also finished in less than one hour since the ObjLanf meta-model already supported multiple data types.
The second extension involved more effort to correctly synthesize C code.
Finally, the most work has been spent on the last extensions.
In total, the development effort could be estimated to be about 15 person-hours.

\paragraph{Fault tolerance.}

\Todo{Fault tolerance}

\paragraph{Execution time.}

\Todo{Execution time}

\paragraph{Modularity.}

\Todo{Modularity}