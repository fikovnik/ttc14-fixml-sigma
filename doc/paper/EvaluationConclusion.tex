%!TEX root = ttc14-fixml.tex

\section{Evaluation and Conclusion}
\label{sec:EvaluationConclusion}

In this section we evaluate the proposed solution using the evaluation criteria proposed in the case study description~\cite{Lano2014}.
For the brevity, we only include the evaluation of the core problem solution.
However, each extension is represented as a separate project and therefore further evaluation can be derived by comparing the changes required for their respective implementations.

\paragraph{Complexity.}

The case study description proposes to measure the complexity as the number of operator occurrences, features and entity type name references in the specification expressions.

To the best of our knowledge we do not know about any tool that can provide this metric for Scala code.
We can therefore only provide our own estimate which is a rough number since it depends at which depth an expression is considered atomic.
The M2M transformation of the core problem solution contains about 450 expressions. It uses 18 meta-models classes with 23 references.

\paragraph{Accuracy.}
%
The accuracy measures the syntactical correctness of the generated source code and how well does the code represents the \FIXML messages.

The generated code is checked to be compiled for all valid examples provided in the case study.
The compilation does not produce any warning and does not require any special compiler settings\footnote{The C/C++ \texttt{-fPIC} option is a standard option allowing the object files to be included in libraries, \Cf \url{http://gcc.gnu.org/onlinedocs/gcc/Code-Gen-Options.html\#Code-Gen-Options}}.

We used array representation of the same-tag sibling nodes which improves the quality of the code in comparison to numbered fields, and also provides better scalability.
By implementing the first extension, we have further improved the usefulness of the generated code by guessing the appropriate type (\Cf Section~\ref{sec:Extension1}).
While we do not cover the complete XML Schema type system, the solution is extensible and adding a support for a new data type is just a matter of providing the correct mapping into the corresponding languages.
Finally, we have also implemented the generic \FIXML Schema transformation that should result in a complete representation of \FIXML messages in the different languages.

The case study did not require to provide additional features such as property getters and setters, or C/C++ destructors.
As such the usability of the generated code is rather limited.
However, the implementation of these features should be rather straightforward.

\paragraph{Development effort.}

Providing a correct measure of the complete development effort is not easy since by working on the case study we were also improving \SIGMA at the same time.
The solution for the code problem generating only Java code took about 4 person-hours in which about half was spent on designing the right ObjLang model by \emph{trial-and-error}.
Extending the code generator facilities for \Csharp was almost for free as it involves only few lines of code.
The C++ version was ready in less than one person-hour.
The first extension was also finished in less than one hour since the ObjLang meta-model already supported multiple data types.
The second extension involved more effort to correctly synthesize C code.
Finally, the most work has been spent on the last extensions.
In total, the development effort could be estimated to be about 15 person-hours.

\paragraph{Fault tolerance.}
%
A high fault tolerance is in cases that the transformation can detect invalid XML input and produce accurate results.
In our solution, the well-formedness of the XML document is verified by the underlying Scala XML parser that provides a rather verbose error message indicating at which input location it found errors.
Our parser provides additional checks that assert the input document to be a \FIXML 4.4 Schema version and not just any well-formed XML document or \FIXML DTD version.
This check is however rather simple, but could be easily extended by attaching a \FIXML XML Schema location so the parser can verify the document correctness.

\paragraph{Execution time.}

The execution time for the transformations of the test data set provided by the case study is shown in Table~\ref{tab:ExecutionTime}.
The times were collected by running the transformations in the SHARE virtualized environment \texttt{Ubuntu12LTS\_TTC14\_64bit\SIGMA.vdi}.
Therefore the following measures should be considered as micro-benchmark metrics with all the drawback that micro-benchmarking brings.

\begin{table}[h!t]
  \centering
  \begin{tabular}{| l | r | r | r | r |}
    \hline
    \textbf{Message} & \textbf{T2M} & \textbf{M2M} & \textbf{M2T} & \textbf{Total} \\
    \hline
    \texttt{test1}   & 9      & 70     & 191   & 270  \\
    \texttt{test2}   & 27     & 351    & 284   & 662  \\
    \texttt{test5}   & 1201   & 2734   & 459   & 4394 \\
    \texttt{test6}   & 376    & 1387   & 393   & 2156 \\
    \hline
    \textbf{Total}   & 1613   & 4542   & 1327  & 7482 \\
    \hline
  \end{tabular}
  \caption{Execution times for the test \FIXML messages as measured on SHARE (in milliseconds)}
  \label{tab:ExecutionTime}
\end{table}


\paragraph{Modularity.}
%
Modularity is defined as $1 - \frac{d}{r}$, where $d$ is the number of dependencies between rules (implicit or explicit calls, ordering dependencies, inheritance or other forms of control or data dependence) and $r$ is the number of rules~\cite{Lano2014}.

The \Scala|XMLMM2ObjLang| for the core problem consists of $r=8$ rules.
There is one top level rule and all the other rules are calls, explicitly making $d=7$.
The modularity according to the formula is therefore $0.125$.