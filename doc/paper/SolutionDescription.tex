%!TEX root = ttc14-fixml.tex

\enlargethispage{20mm}

\Section{Solution Description}
\label{sec:SolutionDescription}

The core problem of this case study is generating source code from a \FIXML message structure.
The input is a file representing an \FIXML 4.4 message~\cite{FIXML2004} and the output is some Java, \Csharp and C++ sources that represent the structure of the given \FIXML message. 
As suggested, our solution is realized by a systematic model transformation from an XML file to an XML model, which is transformed into an object-oriented language model, from which we serialize into source code.

\vspace*{-5mm}
\paragraph{Prerequisites}
%
In \SIGMA, EMF models are aligned with Scala through automatically generated extension traits (placed in an \href{https://github.com/fikovnik/ttc14-fixml-sigma/blob/master/ttc14-fixml-base/src-gen}{src-gen} folder) that allows for a seamless model navigation and modification using standard Scala expressions.
This includes
%
\begin{inparaitem}[]
  \item omitting \Scala|get| and \Scala|set| prefixes,
  \item convenient first-order logic collection operations (\Eg, \Scala|map|, \Scala|filter|, \Scala|reduce|), and
  \item first-class constructs for creating new model elements.
\end{inparaitem}

\Subsection{\FIXML XML Message to XML Model}

The parsing of an XML document is handled by a Scala library. 
Therefore this tasks in essentially a M2M transformation between the Scala XML model and the XML model specified in the case study description.
This is a trivial operational-style transformation that has been realized by the \href{https://github.com/fikovnik/ttc14-fixml-sigma/blob/master/ttc14-fixml-base/src/fr/inria/spirals/sigma/ttc14/fixml/FIXMLParser.scala}{\Scala|FIXMLParser|} class.

\Subsection{XML Model to ObjLang Model}

\enlargethispage{20mm}

The ObjLang meta-model chosen for this solution originates from the Featherweight Java model~\footnote{\url{http://bit.ly/1mHRkwM}}. It provides a reasonable abstraction for an object-oriented programming language, supporting basic classes, fields and expressions.

In \SIGMA, a M2M transformation is represented as a Scala class that inherits from the \Scala|M2MT| base class.
Concretely, the \href{https://github.com/fikovnik/ttc14-fixml-sigma/blob/master/ttc14-fixml-base/src/fr/inria/spirals/sigma/ttc14/fixml/XMLMM2ObjLang.scala}{\Scala|XMLMM2ObjLang|} our transformation class is defined as:
%
\begin{scalacode}
class XMLMM2ObjLang extends M2MT with XMLMM with ObjLang { // mix-in generated extensions
  // rule definitions 
}  
\end{scalacode}
%
Within the class body, an arbitrary number of transformation rules can be specified as methods using parameters to define the transformation source-target relation.
For example, the first rule of the transformation from \Scala|XMLNode| into a \Scala|Class| is defined as:
%
\begin{scalacode}
def ruleXMLNode2Class(s: XMLNode, t: Class) {
  s.allSameSiblings foreach (associate(_, t))
  t.name = s.tag
  t.members ++= s.sTargets[Constructor]
  t.members ++= s.allAttributes.sTarget[Field]; t.members ++= s.allSubnodes.sTarget[Field]
}
\end{scalacode}
%
This rule represents a matched rule which is automatically applied for all matching elements.
When such a rule is executed, the transformation engine first creates all the defined target elements and then calls the method whose body populates their content using arbitrary Scala code.
A matched rule is applied once and only once for each matching source element, creating a 1:1 or 1:N mapping.
However, in the current scenario, all XML node siblings with the same tag name should be mapped into the same class (N:1 mapping).
This can be done by explicitly associating the siblings to the same class (line 2)\footnote{The \Scala|allSameSiblings| is a helper collecting all same named siblings.}.
%
The \Scala|sTarget(s)| methods are used to relate the corresponding target element(s) that has been already or can be transformed from source element(s).
On lines 4 and 5 the use of these methods will populate the content of the newly created class by in turn executing the corresponding rules, \Ie \Scala|ruleXMLNode2DefaultConstructor|, \Scala|ruleXMLNode2NoneDefaultConstructor|, \Scala|ruleXMLAttribute2Field| and \Scala|ruleXMLNode2Field|.

The last rule converting XML nodes into fields has to handle multiple same-tag siblings.
While the case description proposes to use either multiple fields or a collection, the former brings a scalability problem since in Java, there is a limit of the maximum number of method parameters already exceeded by the test case 5.
Therefore we have opted for the latter and use arrays.
Moreover, even though it has not been specifically requested in the case study, our transformation keeps the default values of the attributes of the different nodes and use them for constructing the instances.
This is done by the \Scala|ruleXMLNode2ConstructorCall| rule.

\Subsection{ObjLang Model to Source code}

\enlargethispage{20mm}

This task involves transforming the ObjLang model into source code.
\SIGMA provides a template-based code-explicit\footnote{It is the output text instead of the transformation code that is escaped.} M2T transformation DSL that relies on Scala support for multi-line string literal and string interpolation.
%
Since we target multiple programming languages, we organize the code generation in a set of Scala classes and use inheritance and class mix-ins to modularly compose configurations for the respective languages.
In the base classes we define methods that synthesize expressions and data types (\href{https://github.com/fikovnik/ttc14-fixml-sigma/blob/master/ttc14-fixml-base/src/fr/inria/spirals/sigma/ttc14/fixml/BaseObjLangMTT.scala}{\Scala{BaseObjLangMTT}}) and abstract a class generation (\href{https://github.com/fikovnik/ttc14-fixml-sigma/blob/master/ttc14-fixml-base/src/fr/inria/spirals/sigma/ttc14/fixml/BaseObjLang2Class.scala}{\Scala{BaseObjLang2Class}}).
Then we use these bases to configure concrete transformations.

The class \href{https://github.com/fikovnik/ttc14-fixml-sigma/blob/master/ttc14-fixml-base/src/fr/inria/spirals/sigma/ttc14/fixml/ObjLang2Java.scala}{\Scala{ObjLang2Java}} contains the Java language specifics and it is almost the same as the \Csharp \href{https://github.com/fikovnik/ttc14-fixml-sigma/blob/master/ttc14-fixml-base/src/fr/inria/spirals/sigma/ttc14/fixml/ObjLang2CSharp.scala}{\Scala{ObjLang2CSharp}} generator.
For C++, the situation is more complicated, since next to the class implementation (\href{https://github.com/fikovnik/ttc14-fixml-sigma/blob/master/ttc14-fixml-base/src/fr/inria/spirals/sigma/ttc14/fixml/ObjLang2CPPClassImpl.scala}{\Scala{ObjLang2CPPClassImpl}}) a header file has to be generated (\href{https://github.com/fikovnik/ttc14-fixml-sigma/blob/master/ttc14-fixml-base/src/fr/inria/spirals/sigma/ttc14/fixml/ObjLang2CPPClassHeader.scala}{\Scala{ObjLang2CPPClassHeader}}).
Furthermore, the ObjLang model is less suitable for C++ classes and thus extra work has to be performed in the M2T transformation.
%
The following is an example of a C++ header generator:
%
\begin{scalacode}
class ObjLang2CPPClassHeader extends BaseObjLang2Class with ObjLang2CPP with ObjLang2CPPHeader {
  override def header = {
    super.header
    source.fields map (_.type_) collect {
      case x: Class => x
    } map (_.cppHeaderFile) foreach { hdr =>
      !s"#include ${hdr.quoted}"
    }
    !endl
  }

  override def genFields = {
    !"public:" indent {
      super.genFields
    }
  }

  override def genConstructors = {
    !"public:" indent {
      super.genConstructors
    }
  }
}
\end{scalacode}
%
It mixes in basic traits and then overrides the template that generates the different segments of the source code.
For example, in C++ we need to add the \Scala|public| keyword before the fields and constructor declarations.
The unary \Scala{!} (bang) operator provides a convenient way to output text.
The \Scala{s} string prefix denotes an interpolated string, which can include Scala expressions.
