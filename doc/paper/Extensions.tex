%!TEX root = ttc14-fixml.tex

\Section{Extensions}
\label{sec:Extensions}

% \enlargethispage{20mm}

In this section we describe our solutions to the case study extensions.
Each extension has been implemented as a separate project on GitHub and therefore an interested reader can easily see what exactly has been changed.

\Subsection{Extension 1 - Selection of Appropriate Data Types}
\label{sec:Extension1}

In this extension we use a simple heuristics to find an appropriate type for a field based on observed attribute values.
We only cover numbers (\Scala|int|, \Scala|long|, \Scala|double|), but since the process is mechanical, it can be easily extended to cover all XML Schema data types.
%
The ObjLang meta-model was extended with new expressions representing the new type literals.
In the M2M transformation we added a function, \Scala|guessDataType|, which uses regular expressions to guess a data type based on a single attribute value.
An attribute can occur multiple times with different values and therefore we need to consider all values at the same time in order to infer a type that is wide enough.
An overridden function \Scala|guessDataType| takes a sequence of attribute values, guesses their individual types and then simply reduces the type to the largest one (\Cf Section~\ref{sec:AppendixGuessDataType}).

\Subsection{Extension 2 - Extension to Additional Languages}
\label{sec:Extension2}

This extension adds a support for the C language.
Since C is not an object-oriented language, more work has to be done in the code generator part, but the M2M transformations remain untouched.
%
Instead of classes, we generate C structs with appropriate functions simulating object constructors (\Cf Section~\ref{sec:AppendixGeneratingCCode}).
In order to simplify the code generator, we use a helper function that allows us to initialize arrays using simple expressions, which is not directly supported by the C language or by its standard library.
%
Despite the lack of object orientation, our organization of the M2T transformation templates makes the implementation only 36 lines longer than the C++ version (\Cf \href{https://github.com/fikovnik/ttc14-fixml-sigma/blob/master/ttc14-fixml-extension-2/src/fr/inria/spirals/sigma/ttc14/fixml/ObjLang2CClassHeader.scala}{\Scala|ObjLang2CClassHeader|} and \href{https://github.com/fikovnik/ttc14-fixml-sigma/blob/master/ttc14-fixml-extension-2/src/fr/inria/spirals/sigma/ttc14/fixml/ObjLang2CClassHeader.scala}{\Scala|ObjLang2CClassImpl|}).

\Subsection{Extension 3 - Generic Transformation}
\label{sec:Extension3}

\enlargethispage{20mm}

Essentially, a generic \FIXML Schema to ObjLang model transformation means creating a generic XML schema to ObjLang transformation.
Such a task is far from being trivial and it would require a significant engineering effort.
Therefore we have chosen an alternative solution in which we transform Java classes generated from an XML schema by the JAXB tool\footnote{Java Architecture for XML Binding \url{https://jaxb.java.net/}}.
The advantage of this solution is that the JAXB already does all the hard work of parsing XML schema, resolving the element inheritance, substitutability, data types and others.
The resulting Java classes in fact represents an object-oriented model and therefore the actual M2M transformation into ObjLang is rather straight forward.
Finally, the JAXB can be thought of as a another model transformation. Therefore our solution is still within the model-driven engineering domain while demonstrating a strong advantage of internal DSL in reusing very easily another API with the host language.

The new input is a location of the \FIXML XSD files and the new transformation workflow consists of 
\begin{inparaenum}[(1)]
	\item XSD to Java sources using JAXB,
	\item Java source to Java classes using a Java compiler,
	\item Java classes to ObjLang,
	\item ObjLang to source source.
\end{inparaenum}
%
The ObjModel had to be extended to cover enumerated types a notion of inheritance and abstract classes.
%
The new transformation (\href{https://github.com/fikovnik/ttc14-fixml-sigma/blob/master/ttc14-fixml-extension-3/src/fr/inria/spirals/sigma/ttc14/fixml/Java2ObjLang.scala}{\Scala|Java2ObjLang|}) is about 30\% smaller than the original transformation and arguably less complex.
It also demonstrates \SIGMA support for manipulating different models than EMF, \Ie, Java model implemented using Java reflection (\href{https://github.com/fikovnik/ttc14-fixml-sigma/blob/master/ttc14-fixml-extension-3/src/fr/inria/spirals/sigma/ttc14/fixml/javamm/JavaClassModel.scala}{\Scala|JavaClassModel|}).
The M2T transformation remained mostly untouched apart from the model extensions.
It is important to note that the C code generator supports neither class inheritance nor abstract classes\footnote{A potential, and heavy, solution is described  by Schreiner~\cite{Schreiner1993}, but is out of the scope of this \TTC case.}.
