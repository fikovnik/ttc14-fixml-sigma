%!TEX root = ttc14-fixml.tex

\enlargethispage{20mm}

\section{Introduction}
\label{sec:Introduction}

%% Overview
In this paper we describe the solution for the \TTC \FIXML case study~\cite{Lano2014} using the \SIGMA internal DSLs~\cite{Krikava2014}.
The case study involves \emph{text-to-model} (T2M), \emph{model-to-model} (M2M) and \emph{model-to-text} (M2T) transformations, generating Java, \Csharp, C++ code from a \FIXML XML messages.
Next to the core task we also solve the three proposed extensions for determining appropriate types of element attributes, generating C code and generic \FIXML schema transformation.

%% Scala
The solution was developed in \SIGMA\footnote{An open source project available on Github \url{https://fikovnik.github.io/Sigma}}, a family of Scala~\cite{Odersky2004} internal DSLs for model manipulation tasks such as model validation, M2M and M2T transformations.
Scala is a statically typed production-ready \emph{General-Purpose Language} (GPL) that supports both object-oriented and functional styles of programming.
It uses type inference to combine static type safety with a \emph{``look and feel''} close to dynamically typed languages.
It is interoperable with Java and has been designed to host internal DSLs~\cite{Chafi2010}.
Furthermore, it is supported by the major integrated development environments.

%% SIGMA
\SIGMA DSLs are embedded in Scala as a library allowing one to manipulate models using high-level constructs similar to ones found in the external model manipulation DSLs such as ETL~\cite{Kolovos2008a} or ATL~\cite{Jouault2006}.
The intent is to provide an approach that developers can use to implement many of the practical model manipulations within a familiar environment, reduced learning overhead and improved usability.

The solution uses the \emph{Eclipse Modeling Framework} (EMF)~\cite{EMF}, which is a popular meta-modeling framework widely used in both academia and industry.
It is directly supported by \SIGMA, however, other meta-modeling frameworks could be used as well, since \SIGMA transformations are technologically agnostic.
It is available on a Github\footnote{\url{https://github.com/fikovnik/ttc14-fixml-sigma}} as well as in the SHARE environment\footnote{\url{http://share20.eu/?page=ConfigureNewSession&vdi=Ubuntu12LTS_TTC14_64bit_SIGMA.vdi}}.

%% Organization
The next section described the solution to the case study core problem, Section~\ref{sec:Extensions} presents the solutions for the extensions and Section~\ref{sec:EvaluationConslusion} evaluates the solution using the evaluation criteria proposed in the case study and concludes the paper.