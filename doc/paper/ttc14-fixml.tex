\documentclass[submission]{eptcs}
\providecommand{\event}{TTC 2014} % Name of the event you are submitting to

\usepackage[usenames,dvipsnames,table]{xcolor}
\usepackage[T1]{fontenc}
\usepackage{geometry}
\usepackage{paralist}
\usepackage{graphicx}
\usepackage[cache]{minted}
\usepackage{url}
\usepackage[utf8]{inputenc}
\usepackage{paralist}
\usepackage{amstext}
\usepackage{amsfonts}
\usepackage{xspace}
\usepackage{todonotes}
\usepackage{hyperref}

\usepackage[skip=0pt]{caption}
\usepackage{setspace}
\setstretch{0.96}

\hypersetup{
  colorlinks, 
  breaklinks, 
  bookmarksnumbered=true,
  pdftitle={Solving the TTC'14 FIXML Case Study with SIGMA},
  pdfauthor={Filip Krikava and Philippe Collet}, 
  linkcolor=Blue, 
  citecolor=BrickRed, 
  filecolor=Blue, 
  urlcolor=Blue
  }

\renewcommand{\ttdefault}{pcr}

\newcommand{\SIGMA}{\textsc{Sigma}\xspace}
\newcommand{\FIXML}{FIXML\xspace}
\newcommand{\TTC}{TTC'14\xspace}
\newcommand{\Csharp}{C\#\xspace}
\newcommand{\Ie}{\emph{i.e.}\xspace}
\newcommand{\Eg}{\emph{e.g.}\xspace}
\newcommand{\Etal}{\emph{et al.}\xspace}
\newcommand{\Cf}{\emph{cf.}\xspace}

\newcommand{\Todo}[1]{\todo[inline]{#1}}
\renewcommand{\Todo}[1]{}

\newminted{scala}{fontsize=\fontsize{8}{8},linenos,numbersep=5pt,frame=lines,framesep=2mm}
\newminted{xml}{fontsize=\fontsize{8}{8},linenos,numbersep=5pt,frame=lines,framesep=2mm}
\newminted{java}{fontsize=\fontsize{8}{8},linenos,numbersep=5pt,frame=lines,framesep=2mm}
\newminted{c}{fontsize=\fontsize{8}{8},linenos,numbersep=5pt,frame=lines,framesep=2mm}
\newmintinline{xml}{fontsize=\fontsize{8}{8}}
\newmintinline{scala}{fontsize=\fontsize{8}{8}}
\newmintinline{c}{fontsize=\fontsize{8}{8}}

\newcommand{\Scala}{\scalainline}

\title{Solving the \TTC \FIXML Case Study with SIGMA}

\author{
  Filip Křikava
  \institute{University Lille 1 - LIFL, France}
  \institute{INRIA Lille, Nord Europe}
  \email{\href{mailto:filip.krikava@inria.fr}{filip.krikava@inria.fr}}
\and
  Philippe Collet
  \institute{Université Nice - Sophia Antipolis, France}
  \institute{CNRS, I3S, UMR 7271}
  \email{\quad \href{mailto:philippe.collet@unice.fr}{philippe.collet@unice.fr}}
}

\def\titlerunning{Solving the \TTC \FIXML Case Study with SIGMA}
\def\authorrunning{F. Křikava and P. Collet}

\begin{document}
\maketitle

\begin{abstract}
In this paper we describe a solution for the \emph{Transformation Tool Contest 2014} (\TTC) FIXML case study using \SIGMA, a family of Scala internal \emph{Domain-Specific Languages} (DSLs) for model manipulation that provides expressive and efficient API for model consistency checking and model transformations.
The solution solves the core transformation task as well as all the three extensions.
\end{abstract}

%!TEX root = ttc14-fixml.tex

\section{Introduction}
\label{sec:Introduction}

%% Overview
In this paper we describe the solution for the \TTC \FIXML case study~\cite{Lano2014} using the \SIGMA internal DSLs~\cite{Krikava2014}.
The case study involves \emph{text-to-model} (T2M), \emph{model-to-model} (M2M) and \emph{model-to-text} (M2T) transformations, generating Java, \Csharp, C++ code from a \FIXML XML messages.
Furthermore, we describe our approach to the three proposed extensions.
The complete solution is available on a Github\footnote{\url{https://github.com/fikovnik/ttc14-fixml-sigma}} as well as in the SHARE\footnote{\url{http://is.ieis.tue.nl/staff/pvgorp/share/}} environment in the virtual machine image \texttt{Ubuntu12LTS\_TTC14\_64bit\_SIGMA.vdi}.

%% SIGMA
The solution is developed in \SIGMA, a family of Scala\footnote{\url{http://scala-lang.org/}} internal DSLs for model manipulation tasks such as model validation and model transformations.
Developed as an open source project hosted on Github~\footnote{\url{https://fikovnik.github.io/Sigma}},
\SIGMA is a library that provides a dedicated Scala API allowing to manipulate models using high-level constructs similar to ones found in the external model manipulation DSLs such as ETL~\cite{Kolovos2008a} or ATL~\cite{Jouault2006}.
The intent is to provide an approach that developers can use to implement many of the practical model manipulations within a familiar environment, reduced learning overhead and improved usability.

The solution uses the \emph{Eclipse Modeling Framework} (EMF)~\cite{EMF}, which is a popular meta-modeling framework widely used in both academia and industry.
It is directly supported by \SIGMA, however, other meta-modeling frameworks could be used as well, since \SIGMA transformations are technologically agnostic.

%% Organization
The remainder of this document is organized as follows:
\begin{inparaitem}[]
	\item Section~\ref{sec:SigmaOverview} gives a brief overview of \SIGMA.
	\item Section~\ref{sec:SolutionDescription} describes the solution for the case study core problem.
	\item Section~\ref{sec:Extensions} presents the solutions for the three case study extensions.
	\item Section~\ref{sec:Evaluation} evaluates the solution using the evaluation criteria proposed in the case study, and finally Section~\ref{sec:Conclusion} concludes the paper.
\end{inparaitem}
%
% Additionally, two appendixes are provided.
% \begin{inparaitem}[]
% 	\item Appendix~\ref{sec:Configuration} describes the configuration of the SHARE environment that has been done in order to run the \SIGMA solution and
% 	\item Appendix~\ref{sec:Example} shows the generated Java, \Csharp, C++ and C code resulting from running the solution for the example \texttt{test2.xml} \FIXML message.
% \end{inparaitem}
%!TEX root = ttc14-fixml.tex

\section{Solution Description}
\label{sec:SolutionDescription}

This section describes the solution for the core problem of transforming \FIXML messages into source code.
The input is a file representing an \FIXML 4.4 message~\cite{FIXML2004} and the output is a corresponding Java, \Csharp and C++ source representing the given \FIXML message. 
As suggested, the solution is realized by a systematic model transformation from a FIXML message to XML model to an object-oriented language model to a source code.

\vspace*{-5mm}
\paragraph{Prerequisites}
%
In \SIGMA, EMF models are aligned with Scala through automatically generated extension traits\footnote{The generated code is in the \href{https://github.com/fikovnik/ttc14-fixml-sigma/blob/master/ttc14-fixml-base/src-gen}{src-gen} folder} that allows for a seamless model navigation and modification using standard Scala expressions.
This includes
%
\begin{inparaitem}[]
  \item omitting \Scala|get| and \Scala|set| prefixes,
  \item convenient first-order logic collection operations (\Eg, \Scala|map|, \Scala|filter|, \Scala|reduce|), and
  \item first-class constructs for creating new model elements.
\end{inparaitem}

\subsection{\FIXML XML Message to XML Model}

Parsing of an XML document is handled by a Scala library and therefore this tasks in essentially a M2M transformation between Scala XML model and the XML model specified in the case study description (\Cf Figure~\ref{fig:XMLMetaModel}).
This is a trivial operational-style transformation that has been realized by the \href{https://github.com/fikovnik/ttc14-fixml-sigma/blob/master/ttc14-fixml-base/src/fr/inria/spirals/sigma/ttc14/fixml/FIXMLParser.scala}{\Scala|FIXMLParser|} class.

\subsection{XML Model to ObjLang Model}

The ObjLang meta-model chosen for this solution is shown in Figure~\ref{fig:ObjLangMetaModel}.
Originating from the Featherweight Java model~\cite{Igarashi2001}, it provides a reasonable abstraction for an object-oriented programming language, supporting classes, fields and expressions used for field initializations.

In \SIGMA, a M2M transformation is represented as a Scala class that inherits from the \Scala|M2MT| base class, which itself brings M2M DSL constructs into the class scope.
Concretely, the \href{https://github.com/fikovnik/ttc14-fixml-sigma/blob/master/ttc14-fixml-base/src/fr/inria/spirals/sigma/ttc14/fixml/XMLMM2ObjLang.scala}{\Scala|XMLMM2ObjLang|} class is defined as:
%
\begin{scalacode}
class XMLMM2ObjLang extends M2MT with XMLMM with ObjLang { // mix -in generated extensions
  sourceMetaModels = _xmlmm; targetMetaModels = _objlang // set source and target models
}  
\end{scalacode}
%
Within the class body, an arbitrary number of transformation rules can be specified as methods using parameters to define the transformation source-target relation.
For example, the first rule of the transformation from \Scala|XMLNode| into a \Scala|Class| is defined as:
%
\begin{scalacode}
def ruleXMLNode2Class(s: XMLNode, t: Class) {
  s.allSameSiblings foreach (associate(_, t))
  t.name = s.tag
  t.members ++= s.sTargets[Constructor]
  t.members ++= s.allAttributes.sTarget[Field]; t.members ++= s.allSubnodes.sTarget[Field]
}
\end{scalacode}
%
This rule represents a matched rule which is automatically applied for all matching elements.
When such a rule is executed, the transformation engine first creates all the defined target elements and then calls the method whose body populates their content using arbitrary Scala code.
A matched rule is applied once and only once for each matching source element, creating a 1:1 or 1:N mapping.
However, in the current scenario, all XML node siblings with the same tag name should be mapped into the same class (N:1 mapping).
This can be done by associating an explicit association (line 2)\footnote{The \Scala|allSameSiblings| is a helper collecting all same named siblings.}.
%
The \Scala|sTarget(s)| methods relate the corresponding target element(s) that has been already or can be transformed from source element(s).
On lines 4 and 5 the use of these methods will populate the content of the newly created class by in turn executing corresponding rules \Scala|ruleXMLNode2DefaultConstructor|, \Scala|ruleXMLNode2NoneDefaultConstructor|, \Scala|ruleXMLAttribute2Field| and \Scala|ruleXMLNode2Field| (see Listings in~\ref{sec:TransformationRules}).

The last rule converting XML nodes into fields has to handle multiple same-tag siblings.
While the case description proposes to use either multiple fields or a collection, the former brings a scalability problem since in Java, there is a limit of the maximum number of method parameters already exceeded by the test case 5.
Therefore we have opted for the latter and use arrays.
Moreover, even though it has not been specifically requested in the case study, our transformation keeps the default values of the attributes of the different nodes and use them for constructing the instances.
This is done by the \Scala|ruleXMLNode2ConstructorCall| rule (\Cf Section~\ref{ConstructorArguments}).

\subsection{ObjLang Model to Source code}

This task involves transforming the ObjLang model into source code.
\SIGMA provides a template-based code-explicit\footnote{It is the output text instead of the transformation code that is escaped} M2T transformation DSL, relying Scala multi-line string literals and string interpolations.
%
Since we target multiple programming languages, we organize the code generation in a set of Scala classes and use inheritance and class mix-ins to modularly compose configuration for the respective languages (\Cf Section~\ref{sec:M2TClassHierarchy}).
In the base classes we define methods that synthesize expressions and data types (\href{https://github.com/fikovnik/ttc14-fixml-sigma/blob/master/ttc14-fixml-base/src/fr/inria/spirals/sigma/ttc14/fixml/BaseObjLangMTT.scala}{\Scala{BaseObjLangMTT}}) and abstract a class generation (\href{https://github.com/fikovnik/ttc14-fixml-sigma/blob/master/ttc14-fixml-base/src/fr/inria/spirals/sigma/ttc14/fixml/BaseObjLang2Class.scala}{\Scala{BaseObjLang2Class}}).

The class \href{https://github.com/fikovnik/ttc14-fixml-sigma/blob/master/ttc14-fixml-base/src/fr/inria/spirals/sigma/ttc14/fixml/ObjLang2Java.scala}{\Scala{ObjLang2Java}} contains the Java language specifics.
The generator for the \Csharp is exactly the same and the class \href{https://github.com/fikovnik/ttc14-fixml-sigma/blob/master/ttc14-fixml-base/src/fr/inria/spirals/sigma/ttc14/fixml/ObjLang2CSharp.scala}{\Scala{ObjLang2CSharp}} only redefines the \Scala{string} data type.
For C++, the situation is a bit more complicated since also a header file has to be generated and the ObjLang model is less suitable for C++ classes and thus more work has to be performed in the M2T transformation.
More methods have to be overwritten in the transformations \href{https://github.com/fikovnik/ttc14-fixml-sigma/blob/master/ttc14-fixml-base/src/fr/inria/spirals/sigma/ttc14/fixml/ObjLang2CPPClassHeader.scala}{\Scala{ObjLang2CPPClassHeader}} and \href{https://github.com/fikovnik/ttc14-fixml-sigma/blob/master/ttc14-fixml-base/src/fr/inria/spirals/sigma/ttc14/fixml/ObjLang2CPPClassImpl.scala}{\Scala{ObjLang2CPPClassImpl}} making the C++ code generator 88 lines longer than Java or \Csharp ones.
% %!TEX root = ttc14-fixml.tex

\section{Extensions}
\label{sec:Extensions}

In this section we describe our solutions to the three proposed case study extensions.

\subsection{Extension 1 - Selection of Appropriate Data Types}
\label{sec:Extension1}

The source concerning this first extension is located in the \href{https://github.com/fikovnik/ttc14-fixml-sigma/tree/master/ttc14-fixml-extension-1}{\texttt{ttc14-fixml-extension-1}} directory.

\bigskip

There are three changes needed in order to implement this extension.
\begin{itemize}[(1)]
	\item The first one is in the ObjLang meta-model where we need to add new expression classes representing literals for the new data types.
	In this extension we consider the following new data types: \Scala{double}, \Scala{long} and \Scala{integer}.
	While this list does not cover all the possible XML Schema data types, it provides a good basis for demonstrating how some support for additional ones could be added.

	\item The second one concerns the M2M transformation.
	We have to add the necessary support for guessing the data type of an attribute based on the string values of all of the same-tag siblings that have the attribute in question.
	Following is the code snippet that realizes it:
	%
	\begin{scalacode}
	  // basic types
	  val DTString = DataType(name = "string")
	  val DTDouble = DataType(name = "double")
	  val DTLong = DataType(name = "long")
	  val DTInteger = DataType(name = "int")

	  // it also stores the promotion ordering from right to left
	  val Builtins = Seq(DTString, DTDouble, DTLong, DTInteger)

	  private val PDouble = """([+-]?\d+.\d+)""".r
	  private val PInteger = """([+-]?\d+)""".r

	  def guessDataType(value: String): DataType = value match {
	    case PDouble(_) => DTDouble
	    case PInteger(_) => Try(Integer.parseInt(value)) map (_ => DTInteger) getOrElse (DTLong)
	    case _ => DTString
	  }

	  def guessDataType(values: Seq[String]): DataType =
	    values map guessDataType reduce { (a, b) =>
	      if (Builtins.indexOf(a) < Builtins.indexOf(b)) a else b
	    }
	\end{scalacode}

	\item Finally, we need to update the code transformers to generate the appropriate data types.
	For example, in the C++ generator:
	\begin{scalacode}
override def class2Code(p: DataType) = {
  import XMLMM2ObjLang._
  p match {
    case DTString => "std::string"
    case DTDouble => "double"
    case DTLong => "long"
    case DTInteger => "int"
  }
}		
	\end{scalacode}
\end{itemize}

\subsection{Extension 2 - Extension to Additional Languages}
\label{sec:Extension2}

This extension adds a support for the C language.
Since C is not an object-oriented language, more work have to be done in the code generator part.
It is important to note, that the M2M or T2M transformations have to remain untouched.

Instead of classes, in the case of the C language, we generate C struct declarations with appropriate functions simulating object constructors.
For each ObjLang class we generate a header file with a struct definition, a function for the struct creation and a set of functions representing class constructors.
Following is an example of the C code synthesized from this \FIXML message (for the \Scala|Pty| node):
%
\inputminted[fontsize=\fontsize{8}{8},linenos,numbersep=5pt,frame=lines,framesep=2mm]{xml}{listings/example-for-c-code.xml}
%
\begin{ccode}
#ifndef _Pty_H_
#define _Pty_H_

#include <stdlib.h>

#include "Sub.h"

typedef struct {
  char* _R;
  char* _ID;
  Sub** Sub_objects;
} Pty;

Pty* Pty_new();
Pty* Pty_init_custom(Pty* this, char* _R, char* _ID, Sub** Sub_objects);
Pty* Pty_init(Pty* this);

#endif // _Pty_H_	
\end{ccode}
%
\begin{ccode}
#include "arrays.h"

#include "Pty.h"

Pty* Pty_new() {
  return (Pty*) malloc(sizeof(Pty));
}

Pty* Pty_init_custom(Pty* this, char* _R, char* _ID, Sub** Sub_objects) {
  this->_R = _R;
  this->_ID = _ID;
  this->Sub_objects = Sub_objects;
  return this;
}

Pty* Pty_init(Pty* this) {
  this->_R = "21";
  this->_ID = "OCC";
  this->Sub_objects = (Sub**) new_array(2, Sub_init_custom(Sub_new(), "2", "ZZZ"), NULL);
  return this;
}
\end{ccode}

In order to simplify the code generator, we use a helper function \cinline{void **new_array(int size, ...)} that allows us to initialize arrays using simple expressions, which is not directly supported by C language or by the standard C library.

Despite the fact that C is not object-oriented, our organization of the M2T transformation templates makes the implementation only 36 lines longer (30\%) than the C++ version.
The source concerning the extension 2 solution can be found in the \href{https://github.com/fikovnik/ttc14-fixml-sigma/tree/master/ttc14-fixml-extension-2}{\texttt{ttc14-fixml-extension-2}} directory.

\subsection{Extension 3 - Generic Transformation}
\label{sec:Extension3}

The last extension aims at generic transformation between \FIXML Schema and ObjLang model.
Providing such a generic transformation essentially means creating a generic XML schema to ObjLang transformation.
Such a task is far from being trivial and it would require a significant engineering effort.
Therefore we have decided for an alternative solution in which we transform Java classes generated from an XML schema by the Java Architecture for XML Binding (JAXB) tool\footnote{\url{https://jaxb.java.net/}}.

The advantage of this solution is that the JAXB already does all the hard work of parsing XML schema, resolving the element inheritance, substitutability, data types and others.
The model of Java classes in an object-oriented model and therefore the actual transformation into ObjLang is actually easier than in the case of the \FIXML messages.
Finally, the JAXB can be thought of as a another model transformation and therefore our solution is still within the model-driven engineering domain.

The new input is a location of the \FIXML XSD files and the new transformation workflow consists of the following stages:
\begin{enumerate}[(1)]
	\item XSD to Java sources using JAXB.
	\item Compilation of Java source using regular Java compiler.
	\item Java model to ObjLang transformation.
	\item ObjLang to source transformation.
\end{enumerate}

\bigskip

The implementation involves following changes:
\begin{enumerate}[(1)]
	\item Extending the ObjLang model with a new classifier representing enumerated types.
	\item Extending the ObjLang model with the notion of a simple class inheritance.
	\item Extending the ObjLang model with the notion of abstract classes.
	\item A new M2M transformation between Java class model represented by the Java reflection API and ObjLang model.
	\item Extending the M2T transformation to cover the new ObjLang model concepts.
\end{enumerate}

The new transformation is about 30\% smaller than the one developed in the first extension and arguably less complex.
It also demonstrates \SIGMA support for manipulating different models than EMF, as well as easily reusing existing libraries.
This is typically the kind of benefits that one can get from \SIGMA, \textit{.i.e.}, being able to easily reuse an existing API (JAXB), while developing transformation rules in a domain-specific environment.
The source concerning the extension 3 solution is in the \href{https://github.com/fikovnik/ttc14-fixml-sigma/tree/master/ttc14-fixml-extension-3}{\texttt{ttc14-fixml-extension-3}} directory.

One limitation of the current implementation is that we do not extend the C code generator to support the notion of class inheritance and for abstract classes.
A potential solution is described by Schreiner~\cite{Schreiner1993}.
However it requires to build a lot of supporting infrastructure which is out of the scope of this \TTC case.
Similarly to extension 1, we do not handle all the XML schema data types, but only the ones that appear in the \FIXML schema.

% %!TEX root = ttc14-fixml.tex
\vspace*{-3mm}
\section{Evaluation and Conclusion}
\label{sec:EvaluationConclusion}

\enlargethispage{7mm}

We evaluate our solution to the core problem using the evaluation criteria proposed in the case study description~\cite{Lano2014}.

\begin{compactitem}[$-$]
  \item The \emph{complexity} as the number of operator occurrences, features and entity type name references in the specification expressions.
  To the best of our knowledge there is no tool providing this metric for Scala code.
  We therefore only provide our own estimate for the M2M transformation, which contains about 450 expressions and uses 18 meta-models classes with 23 references.

  \item The \emph{accuracy} measures the syntactical correctness of the generated source code and how well the code represents the \FIXML messages.
  The generated code compiles for all languages without any warning nor any special compiler settings.
  Using arrays to represent same-tag sibling nodes improves the quality and scalability of the code which is further enhanced by data type heuristics for field types.
  Finally, we have also implemented the generic \FIXML Schema transformation that should result in a complete representation of \FIXML messages in the different languages.

  \item The development effort is estimated to be about 15 person-hours for the core problem.

  \item The \emph{fault tolerance} is high since the Scala XML library can detect invalid XML with accurate parsing errors.

  \item For all test cases (1, 2, 5 and 6), the \emph{execution time} is about 7500ms for all the transformations on SHARE.

  \item \emph{Modularity} for the M2M transformation is $1 - \frac{d}{r} = \frac{7}{8} = 0.125$, where $d$ is the number of dependencies between rules and $r$ is the number of rules.

  \item The level of \emph{abstraction} for both the M2M and M2T transformations is medium since the rules are defined declaratively (high abstraction), but their content is an imperative code (medium).

\end{compactitem}

Despite that we opted for a complex ObjLang model, the resulting transformations are rather expressive and quite concise.
The complete implementation of the core problem consists of 500 lines of Scala code\footnote{The extension 1 consists of 550, extension 2 of 720 and extension 3 of 770 source lines of code.}.
This \FIXML case study provides a good illustration for some of the capabilities of an internal DSL approach to model manipulations in the model-driven engineering domain.











\paragraph{Acknowledgment}
This work is partially supported by the Datalyse project \url{www.datalyse.fr}.

\bibliographystyle{eptcs}
\bibliography{references.bib}	

\appendix

%!TEX root = ttc14-fixml.tex

\section{Meta-Models}
\label{sec:MetaModels}

\enlargethispage{20mm}

\subsection{XML Meta-Model}

The XML model specified in the case study description~\cite{Lano2014}.

\begin{figure}[h!bt]
  \centering
  \includegraphics[width=.6\textwidth]{figures/XMLMetaModel.pdf}
  \caption{XML meta-model}
  \label{fig:XMLMetaModel}
\end{figure}

\subsection{ObjLang Meta-Model}

The meta-model representing an object oriented language originating from the Featherweight Java model~\cite{Igarashi2001} (concretely from the version available at the EMFtext website\footnote{\url{http://www.emftext.org/index.php/EMFText_Concrete_Syntax_Zoo_Featherweight_Java}}).

\begin{figure}[h!bt]
  \centering
  \includegraphics[width=\textwidth]{figures/ObjLangMetaModel.pdf}
  \caption{ObjLang meta-model}
  \label{fig:ObjLangMetaModel}
\end{figure}


%!TEX root = ttc14-fixml.tex

\section{Listings}

\subsection{Transformation Rules}
\label{sec:TransformationRules}

\begin{scalacode}
def ruleXMLNode2DefaultConstructor(s: XMLNode, t: Constructor) {
  s.allSameSiblings foreach (associate(_, t))
}
\end{scalacode}

\begin{scalacode}
def ruleXMLNode2NonDefaultConstructor(s: XMLNode, t: Constructor) = guardedBy {
  !s.isEmptyLeaf
} transform {

  s.allSameSiblings foreach (associate(_, t))

  for (e <- (s.allAttributes ++ s.allSubnodes.distinctBy(_.tag))) {
    val param = e.sTarget[Parameter]
    val field = e.sTarget[Field]

    t.parameters += param
    t.initialisations += FieldInitialisiation(
      field = field,
      expression = ParameterAccess(parameter = param))
  }
}
\end{scalacode}

\begin{scalacode}
def ruleXMLAttribute2ConstructorParameter(s: XMLAttribute, t: Parameter) {
  t.name = checkName(s.name)
  t.type_ = s.sTarget[Field].type_
}
\end{scalacode}

\begin{scalacode}
def ruleXMLNode2ConstructorParameter(s: XMLNode, t: Parameter) {
  val field = s.sTarget[Field]

  t.name = field.name
  t.many = field.many
  t.type_ = field.type_
}
\end{scalacode}

\begin{scalacode}
@LazyUnique
def ruleXMLAttribute2Field(s: XMLAttribute, t: Field) {
  t.name = checkName(s.name)

  t.type_ = DTString
  t.initialValue = StringLiteral(s.value)
}
\end{scalacode}

\begin{scalacode}
@LazyUnique
def ruleXMLNode2Field(s: XMLNode, t: Field) {
  val allSiblings = s.allSameSiblings
  allSiblings foreach (associate(_, t))

  t.type_ = s.sTarget[Class]

  val groups = (s +: allSiblings) groupBy (_.eContainer)
  val max = groups.values map (_.size) max

  if (max > 1) {
    t.name = s.tag + "_objects"
    t.many = true
    val init = ArrayLiteral(type_ = s.sTarget[Class])
    val siblings = groups(s.eContainer)
    
    init.elements ++= siblings.sTarget[ConstructorCall]
    init.elements ++= 0 until (max - siblings.size) map (_ => NullLiteral())
    t.initialValue = init
  } else {
    t.name = s.tag + "_object"
    t.initialValue = s.sTarget[ConstructorCall]
  }
}  
\end{scalacode}

\begin{scalacode}
@Lazy
def ruleXMLNode2ConstructorCall(s: XMLNode, t: ConstructorCall) {
  val constructor = s.sTargets[Constructor]
    .find { c =>
      (c.parameters.isEmpty && s.isEmptyLeaf) ||
      (c.parameters.nonEmpty && !s.isEmptyLeaf)
    }
    .get

  t.constructor = constructor

  t.arguments ++= {
    for {
      param <- constructor.parameters
      source = param.sSource.get
    } yield {
      source match {
        case attr: XMLAttribute =>
          // we can cast since attributes have always primitive types
          val dataType = param.type_.asInstanceOf[DataType]

          s.attributes
            .find(_.name == attr.name)
            .map { local => StringLiteral(local.value) }
            .getOrElse(NullLiteral())

        case node: XMLNode =>
          s.subnodes.filter(_.tag == node.tag) match {

            case Seq() if !param.many =>
              NullLiteral()
            case Seq(x) if !param.many =>
              x.sTarget[ConstructorCall]
            case Seq(xs @ _*) =>
              val groups = (node +: node.allSameSiblings) groupBy (_.eContainer)
              val max = groups.values map (_.size) max
              
              val init = ArrayLiteral(type_ = param.type_)
              init.elements ++= xs.sTarget[ConstructorCall]
              init.elements ++= 0 until (max - xs.size) map (_ => NullLiteral())
              init
          }
      }
    }
  }
}
\end{scalacode}
%!TEX root = ttc14-fixml.tex

\section{Handling Constructor Arguments}
\label{sec:ConstructorArguments}

The number of same-tag sibling nodes can vary within a parent node.
For example:

\inputminted[fontsize=\fontsize{8}{8},linenos,numbersep=5pt,frame=lines,framesep=2mm]{xml}{listings/variable-siblings.xml}

The \Scala|Sub| should be represented by an array field and the default initialization of \Scala|PosRpt| should equal to the following (in Java):
%
\begin{javacode}
public Pty[] Pty_objects = new Pty[] { 
  new Pty("OCC", "21", null, new Sub[] { null, null }),
  new Pty("C", "38", null, new Sub[] { new Sub("ZZZ", "2", null), null }),
  new Pty("C", "38", "Q", new Sub[] { new Sub("ZZZ", "2", null), new Sub("ZZZ", "3", "X") }) 
};
\end{javacode}
%
Note that the first and second instances of \Scala|Pty| contains two and one \Scala|null| respectively in the place of missing \Scala|Sub| subnode.

The \Scala|ConstructorCall| used for field initializations in the \Scala|ruleXMLNode2Field| is created from an XML node using the last rule in the transformation:
%
\begin{scalacode}
@Lazy
def ruleXMLNode2ConstructorCall(s: XMLNode, t: ConstructorCall) {
  val constructor = s.sTargets[Constructor]
    .find { c =>
      (c.parameters.isEmpty && s.isEmptyLeaf) ||
      (c.parameters.nonEmpty && !s.isEmptyLeaf)
    }
    .get

  t.constructor = constructor

  t.arguments ++= {
    for {
      param <- constructor.parameters
      source = param.sSource.get
    } yield {
      source match {
        case attr: XMLAttribute =>
          // we can cast since attributes have always primitive types
          val dataType = param.type_.asInstanceOf[DataType]

          s.attributes
            .find(_.name == attr.name)
            .map { local => StringLiteral(local.value) }
            .getOrElse(NullLiteral())

        case node: XMLNode =>
          s.subnodes.filter(_.tag == node.tag) match {

            case Seq() if !param.many =>
              NullLiteral()
            case Seq(x) if !param.many =>
              x.sTarget[ConstructorCall]
            case Seq(xs @ _*) =>
              val groups = (node +: node.allSameSiblings) groupBy (_.eContainer)
              val max = groups.values map (_.size) max
              
              val init = ArrayLiteral(type_ = param.type_)
              init.elements ++= xs.sTarget[ConstructorCall]
              init.elements ++= 0 until (max - xs.size) map (_ => NullLiteral())
              init
          }
      }
    }
  }
}
\end{scalacode}  

First we need to find which constructor shall be used depending whether the given XML node (or any of its same-tag siblings) contains any attributes or subnodes.
Next, we need to resolve the arguments for the case of non-default constructor.
We do this by using the sources, \Ie, the source elements (XML node or XML attribute) that were used to create the constructor parameters.
\SIGMA provides \Scala|sSource| method that is the inverse of \Scala|sTarget| call with the difference that it will not trigger any rule execution.
In the pattern matching we need to cover all possible cases such as an attribute defined locally or an attribute defined in a same-tag sibling, thus using \Scala|null| for its initialization.

%!TEX root = ttc14-fixml.tex

\section{M2T Transformation Class Hierarchy}
\label{sec:M2TClassHierarchy}

\begin{figure}[h!bt]
  \centering
  \includegraphics[width=\textwidth]{figures/M2TClassHierarchy.pdf}
  \caption{M2T transformation class hierarchy including C code generation}
  \label{fig:M2TClassHierarchy}
\end{figure}

\begin{table}
	\centering
  \begin{tabular}{l|l}
  \hline
  \textbf{File name}                  & \textbf{Source line of code} \\ \hline
  ObjLang2C.scala              & 50                  \\
  ObjLang2CClassImpl.scala     & 42                  \\
  ObjLang2CPPClassHeader.scala & 38                  \\
  BaseObjLangMTT.scala         & 35                  \\
  ObjLang2CClassHeader.scala   & 35                  \\
  BaseObjLang2Class.scala      & 33                  \\
  ObjLang2CPP.scala            & 27                  \\
  ObjLang2CPPClassImpl.scala   & 26                  \\
  ObjLang2Java.scala           & 15                  \\
  ObjLang2CSharp.scala         & 15                  \\
  CHeader.scala                & 15                  \\ \hline
  \textbf{Total}                        & \textbf{331}                 \\ \hline
  \end{tabular}
  \caption{Source lines of code for the complete M2T transformation including C code generation}
\end{table}

% \section{Configuring SHARE Environment}
% \label{sec:Configuration}

% \section{Example of \texttt{test2.xml} Output}
% \label{sec:Example}

\end{document}